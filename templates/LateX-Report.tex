\documentclass[11pt]{article}
\usepackage{url}
\usepackage{siunitx}
\usepackage{wrapfig}
\usepackage{mhchem}
\usepackage{amsmath}
%\usepackage{minted} %For Source code uncomment before final compilation
\usepackage{amssymb} %symbols and formlae
\usepackage[margin=1in]{geometry} % set page margins automatically
\usepackage{array} %tables
\usepackage{graphicx} %better images
\graphicspath{ {images/}} %path for images
\usepackage{latexsym} %latex symbols
\usepackage{setspace} %Line Spacing
\usepackage{lmodern}
\usepackage{caption}
\usepackage{subcaption} %caption multi picture
\usepackage{multirow} %Multiple rows as one cell
%For Caption of table to be on top
\usepackage{floatrow}
 \floatsetup[table]{capposition=top}
\usepackage{amsfonts}
\usepackage{booktabs}
\renewcommand{\arraystretch}{1.5}
\usepackage{siunitx}
\usepackage{tabularx}
\usepackage{nicefrac}
\usepackage{float}

%First page
\usepackage{pdfpages}
\newcommand{\degc}{\si{\degree}C}




%
\title{\vspace{-7ex}\textsc{Title}\vspace{-2ex}}
\author{\emph{Tengse, Prasad}\\}
\date{}
%Begin Protocol
\begin{document}
		%\includepdf{titlepages}
		\tableofcontents
		\listoffigures
		\pagebreak
		%end contents page


\section{Introduction}
\hspace{1.5ex}

%USe for minted code should escape from shell
%\usemintedstyle[python]{perldoc}
%\inputminted[breaklines,breakanywhere]{python}{Exercise_12_for_LateX.py}


\begin{thebibliography}{10}
	\bibitem{ref_1}
	Reference 1

	\bibitem{wiki_dpssd}
	Wikipedia \\
	\url{https://en.wikipedia.org/wiki/LateX}\\
	\textit{Accessed on November 28\textsuperscript{th}, 2016 at 08:45pm}


	%\bibitem{einstein}
	%Albert Einstein.
	%\textit{Zur Elektrodynamik bewegter K{\"o}rper}. (German)
	%[\textit{On the electrodynamics of moving bodies}].
	%Annalen der Physik, 322(10):891–921, 1905.

\end{thebibliography}
\pagebreak

%Appendix


\end{document}
